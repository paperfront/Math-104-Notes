\documentclass[11pt, oneside]{article}   	% use "amsart" instead of "article" for AMSLaTeX format       		
% See geometry.pdf to learn the layout options. There are lots.		% ... or a4paper or a5paper or ... 
%\geometry{landscape}                		% Activate for rotated page geometry
%\usepackage[parfill]{parskip}    		% Activate to begin paragraphs with an empty line rather than an indent
\usepackage{graphicx}				% Use pdf, png, jpg, or eps§ with pdflatex; use eps in DVI mode
								% TeX will automatically convert eps --> pdf in pdflatex		
\usepackage{amssymb}
\usepackage{titlesec}
\usepackage{titletoc}
\usepackage{lipsum}

%SetFonts

%SetFonts


% Definitions
\newcommand{\N}{{\mathbb N}}
\newcommand{\Q}{{\mathbb Q}}
\newcommand{\R}{{\mathbb R}}

\usepackage[left=3cm,right=3cm,top=2cm,bottom=2cm]{geometry} % page settings
\usepackage{amsmath} % provides many mathematical environments & tools
\usepackage{amsthm}

\usepackage{tcolorbox}
\tcbuselibrary{theorems}

\setlength{\parindent}{0mm}
\setcounter{tocdepth}{2}

%Definitions and Theorems
\theoremstyle{definition}
\newtcbtheorem[number within=section]{defn}{Definition}%
{colback=green!5,colframe=green!35!black,fonttitle=\bfseries}{thm}

\newtcbtheorem[number within=section]{thm}{Theorem}%
{colback=purple!5,colframe=purple!35!black,fonttitle=\bfseries}{thm}

\begin{document}

\title{Math 104: Introduction to Analysis}
\author{Matthew Lacayo}
\date{\today}
\maketitle

The notes scribed here follow the textbook Elementary Analysis by Kenneth A. Ross. Some additional remarks may be included based on the lecture teachings of Professor Wei Fan. I have slightly reordered some ideas - namely I began my introduction to topology before beginning the discussion of sequences. This is just so that all concepts can be presented in terms of both topology and the real number system.

\newpage

\tableofcontents


\newpage
\section{Introduction}
Dummy text
\subsection{$\N$, $\Q$, and $\R$}
\subsection{Completeness}
\subsection{Constructing $\R$ from $\Q$} 
\subsection{Solved Exercises}


\newpage
\section{A Brief Remark on Topology}
In this section, we will introduce the notion of a metric space. While we will not delve very deep, it is important to start considering how the absolute value distance function we will use on the real numbers can actually be abstracted, and that many results we prove on real numbers are much more general.
\newline
When working with the set of real numbers, we are able to perform much of our analysis through the use of a distance function, namely \(d(x,y) = |x - y|\). As seen in (insert chapter 1 reference), \(d\) satisfies the triangle inequality, which we will make important use of very soon in our discussion of sequences. However, it turns out that many results on the real numbers that involve \(d\) make use of a small set of fundamental properties that the absolute value function demonstrates. As such, we could perform the same analysis with other functions that also demonstrate these same properties so long as we are able to write them out explicitly.  We begin this abstraction process by defining the concepts of a metric, and metric space. 
\newline
\begin{defn}{Metric Space}{metric_space}
\label{defn:metric_space}
Let S be a set and suppose \(d\) is a function defined on all pairs \((x, y)\). We call \(d\) a metric (or distance function) on S iff it satisfies the following conditions:
\begin{flalign}
&d(x,x) = 0 \text{ \(\forall x \in S\)} \\
&d(x,y) > 0\text{ for distinct \(x,y \in S\)} \\
&d(x,y) = d(y,x)  \text{ \(\forall x,y \in S\)} \\
&d(x,z) \leq d(x,y) + d(y,z)\ \forall x,y,z\in S
\end{flalign}

A \emph{metric space} \(M\) is a pair \((S, d)\) where \(d\) is a metric on \(S\). Note that there can be multiple valid metrics on any given set \(S\), as we shall see soon.
\end{defn}

As you read the coming section on sequences, please consider the situations in which it may be possible to replace the operations we perform on \(|x - y|\) with the more general \(d(x,y)\).

%TODO Define d_{std} along with the general metric spaces of (\R^k, d_{std})

%TODO Explain why an open interval, semi open interval, and a closed interval are all valid subsets of R in the metric space (R, d_{std})

\newpage
\section{Sequences and Series}
\subsection{Limits of Sequences}

%TODO Define limits in metric spaces
\begin{defn}{Convergent Sequence in Metric Space}{metric_space_sequence_convergence}
Let \((S, d)\) be a metric space. Let \((a_n)\) be a sequence of points such that each \(a_n \in S\). \newline 
\(\lim_{n\to\infty}a_n = a \in S\) if:
\begin{align*}
  &\textrm{for each \(\epsilon > 0\), there exists \(N > 0\) such that }\\
  &n > N \implies d(a_n, a) < \epsilon
\end{align*}
In this case, we call \((a_n)\) convergent. This is another example of extending ideas we develop on \(\R\) to work on a general metric space \((S, d)\).
\end{defn}



\subsection{Monotone Sequences and Cauchy Sequences}
\subsection{Lim Sup and Lim Inf}
\subsection{Series}

\newpage
\section{Topology Part 2 - Electric Boogaloo}
We will now expand upon the concept of metric spaces that we covered in \ref{defn:metric_space}. The basis of most topological concepts we will cover are derived from the  definitions of open and closed sets in the following section. 
\subsection{Open Sets}
A helpful tool for understanding open and closed sets will be the idea of an 'open ball'. This name will be justified after we have defined what an open set is.
\begin{defn}{Open Ball}{open_ball}
For some metric space \((S, d)\), \(x\in S\), \(r > 0\),
\begin{displaymath}
  B_r(x) := \{y \in S: d(x,y) < r\} \subseteq S
\end{displaymath}
I will sometimes refer to this concept as an 'epsilon ball', since we will often be considering very small values of \(r\). Thus, we can have an equivalent definition by replacing \(r\) with \(\epsilon\).
\end{defn}

\begin{exmp}{}{}
\((\R, d_{std})\): \(B_1(3) = (2,4)\) where \((2,4)\) is the open interval between 2 and 4.
\end{exmp}
\begin{exmp}{}{}
\((\R^2, d_{std})\): \(B_2((0, 0)) = \{(x, y) \in \R^2: x^2 + y^2 < 4\}\). Note that this is a circle of radius 2 centered at the origin.
\end{exmp}

The reason we call this a ball is because of how we can visualize this set in the metric spaces of \((\R^2,d_{std})\) and \((\R^3, d_{std})\). In \((\R^2,d_{std})\), we can visualize the epsilon ball as the circle of radius \(r\) centered at \(x\). Similarly, in \(\R^3\) we can visualize this as a sphere of radius \(r\) centered at \(x\). In general, one can think of the open ball in \(\R^k\) as being a k-dimensional sphere. Another detail to note is that we need not be working with a metric space of the form \((\R^k, d_{std})\) for the open ball to be useful. We simply reference \(\R^k\) in examples since it is easiest to visualize. 
\newline

 

\begin{defn}{Open Set}{open_set}
Let \((S, d)\) be a metric space. Let \(E \subseteq S\). 
\newline
\(E \subseteq S\) is open if for all \(x \in E\), there \textbf{\textit{exists}} some \(r > 0\) such that \(B_r(x) \subseteq E\).

\end{defn}
%TODO insert open set diagram
The important take away from the above definition is that for any point \(x \in E\), we just need to be able to find 1 value of \(r\) that can be as small as we like such that the ball centered at \(x\) of radius \(r\) is contained entirely within \(E\). 
\newline

\begin{exmp}{}{}
\([0, 1] \subseteq \R\) is not open. Consider the epsilon ball \(B_r(0) = (-r, r) \not\subseteq \R\) since \(-r < 0\) and the interval \([0, 1]\) contains no numbers less than 0.
\end{exmp}

\begin{exmp}{}{}
\((0, 1) \subseteq \R\) is open. To show that this is open, consider any point \(x\in (0, 1)\). Your task is to find some \(r\) small enough such that \(B_r(x)\) is entirely contained within \((0, 1)\). Remember that \(r\) can be expressed in terms of \(x\). It may also be helpful to consider the fact that \(B_r(x)= (x - r, x + r)\).
\end{exmp}
\begin{exmp}{}{S_is_open}
Let \((S, d)\) be a metric space. \(S\) is open, and \(\emptyset\) is open (reminder that \(\emptyset\) is the empty set). To see why \(S\) is open, consider any point \(x \in S\). The epsilon ball \(B_r(x)\) can only contain points in defining set of the metric space \(S\), and so \(B_r(x) \subseteq S\). Thus, \(S\) is open. We say that the empty set is trivially open because there are no elements in it, and so no such epsilon balls can be constructed on the empty set.
\end{exmp}
The following theorem will be proven very explicitly to help build some intuition with open sets. Subsequent proofs will not be broken down as deeply.
\newline
\begin{thm}{}{}
Let \((S, d)\) be a metric space. For any \(x \in S\), and for any \(r > 0\), \(B_r(x)\) is an open subset in \(S\).
\newline
\newline 
To prove this statement, we must show that for any \(y \in B_r(x)\), we can find some \(s > 0\) such that \(B_s(y) \subseteq B_r(x)\). We can think of this intuitively as follows: the point \(y\) must be at some distance \(d < r\) away from \(x\). Thus, we know that we can find points between \(y\) and the border of \(B_r(x)\) by considering those points that are of distance \(r - d\) away from \(y\). So, we let \(s = d - r\). Thus, the maximum distance of any point in \(B_s(y)\) from \(x\) will be \(< d + s = d + (r - d) = r\), and so this point will also lie in \(B_r(x)\). Remember that to show \(B_s(y) \subseteq B_r(x)\), it is sufficient to show that \(a \in B_s(y) \implies a \in B_r(x)\). We can formalize our intuition by using the triangle inequality and the following diagram:
	%TODO Insert diagram for showing epsilon ball is an open subset
\begin{proof}
	For any \(y \in S\), define \(D := d(x, y) < r\), and \(s := r - D > 0\). Consider some \(z \in B_s(y)\).
	\begin{equation*}
  		d(x,z) \leq d(x,y) + d(y,z) = D + d(y,z) < D + s = r
	\end{equation*}
	This implies that \(B_s(y) \subseteq B_r(x)\), and since \(y \in S\) is arbitrary, we have shown that \(B_r(x)\) is open.

\end{proof}
\end{thm}

\begin{thm}{}{union_of_open_sets_is_open}

Let \((S, d)\) be a metric space. Let \(U\) be a collection of open sets (of potentially infinite size): \(U = \{u_i : i \in I\}\). Then,
	\begin{equation*}
  		\bigcup_{i \in I}u_i \textrm{ is open.}
	\end{equation*}
In other words, the union of any collection of open sets is an open set.
\newline 

Intuitively, we can consider the fact that for any element x in one of the open sets \(u_i\), there will exist \(r > 0\) such that the epsilon ball of radius \(r\) is also contained in \(u_i\) since it is given that each \(u_i\) is open. Thus, since the epsilon ball is contained in \(u_i\), that same epsilon ball must also be contained in the union of all the \(u_i\). This is true since no elements are lost when unioning a collection of sets. Thus, since an epsilon ball exists for any element in the union of the \(u_i\), it follows by definition that the the union of the \(u_i\) is also an open set.
\begin{proof}
Consider any \(x \in \bigcup_{i \in I}u_i\). We can infer that there exists some \textit{i} such that \(x \in u_i\). Since \(u_i\) is open, there exists \(r > 0\) such that \(B_r(x) \subseteq u_i\). Thus, it follows that \(B_r(x) \subseteq \bigcup_{i \in I}u_i\). Since our choice of \(x\) was arbitrary, this logic must hold for all \(x \in \bigcup_{i \in I}u_i\). Thus, \(\bigcup_{i \in I}u_i\) is an open set.
\end{proof}

\end{thm}
The following is also worth noting: \newline 
\begin{exmp}{}{}
In the metric space \((\R, d_{std})\):
\begin{itemize}
  \item For \(a \leq b \in \R\), the open interval \((a, b)\) is an open set.
  \item For \(a \in \R\), the open intervals \((- \infty, a)\), and \((a, \infty)\) are open sets.
  \item For \(a \leq b \in \R\), the intervals \([a, b)\), \((a, b]\), and \([a, b]\) are \textbf{not} open sets.
\end{itemize}
These results can easily be proven by using similar logic to that of example 3. Try thinking them through!
\end{exmp}



\subsection{Closed Sets}
Naively, one may think that a closed set is simply a set that is not open. However, the definition of a closed set differs slightly from this, which may seem a bit counter intuitive. Instead, we chose to call a set closed if its \textit{complement} is open. This means that it is possible to have a set that is both open and closed, or a set that is not open and not closed (some examples will be discussed later).
\newline

\begin{defn}{Closed Set}{}
Let \((S, d)\) be a metric space. Let \(E \subseteq S\). \newline 
\(E\) is \textbf{closed} if \(E^C\) is \textit{open}.
\newline 

Remember that \(E^C\) is called the \textit{complement} of \(E\), and is defined as \(E^c := \{x \in S : x \not \in E\}\).
\end{defn}
\begin{exmp}{}{}
\([0, 1] \subseteq \R\) is closed. Since a closed set is defined by whether its complement is open, we must consider \([0,1]^C = (- \infty, 0) \cup (1,  \infty)\). As seen in example 5, both \((- \infty, 0)\) and \((1, \infty)\) are open, and by theorem \ref{thm:union_of_open_sets_is_open}, the union of open sets is also an open set. Thus, \([0,1]^C\) is open, and so \([0,1]\) is closed. Similarly, it can be shown that for any \(a<b \in \R\), the closed interval \([a,b]\) is also a closed set.
\end{exmp}
\begin{exmp}{}{}
\((0, 1) \subseteq \R\) is not closed. \((0, 1)^C = (-\infty, 0] \cup [1, \infty)\), which is not open (seen by considering an epsilon ball centered at 0 or 1: \(B_r(0)\) or \(B_r(1)\)). Since \((0, 1)^C\) is not open, \((0, 1)\) is not closed.
\end{exmp}
\begin{exmp}{}{}
\(S\) is both open \textbf{and} closed and \(\emptyset\) is both open \textbf{and} closed. We already showed in example \ref{exmp:S_is_open} that \(S\) is open and \(\emptyset\) is open. Note that \(S^C = \emptyset\) and \(\emptyset^C = S\). Thus, it follows that \(S\) and \(\emptyset\) are also closed.
\end{exmp}
\begin{exmp}{}{limit_point_intro}
\(E = \{ \frac{1}{n} : n \in \N \} \subseteq \R\) is not closed. \begin{proof}
\(0 \not \in E \implies 0 \in E^C\). For any \(r > 0\), there must exist \(x \in E\) such that \(x \in B_r(0)\). This follows because the sequence \((\frac{1}{n})\) converges to 0, and so the sequence will get within \(\epsilon\) of 0 for any \(\epsilon > 0\). In other words, \(B_r(0) \cap E \not = \emptyset\). This implies that \(B_r(0)\) is not entirely contained within \(E^C\), and so \(E^C\) is not open. Thus, \(E\) is not closed.
\end{proof}

\end{exmp}





\subsection{Closure}
Please carefully review example \ref{exmp:limit_point_intro}, as this will motivate our coming discussion of limit points. Note what we observed: we had a sequence contained within \(E\) that converged to a value that was not in \(E\). This led us to conclude that \(E\) was not closed, since any epsilon ball centered at the convergent point must contain a point in \(E\) by definition of convergent sequence. Thus, no epsilon ball centered at the convergent point can be contained entirely inside \(E^C\), which means \(E^C\) cannot be an open set. This shows that \(E\) is not closed. 
\begin{defn}{Limit Point}{limit_point}
Let \((S, d)\) be a metric space. Let \(E \subseteq S\). We say \(x \in S\) is a \textit{limit point} of E if for all \(r > 0\), \(B_r(x) \cap E\) contains some point \(y \in S\) such that \(x \not = y\). Here are some equivalent formulations of this idea:
	\begin{itemize}
  		\item \(E\) contains points arbitrarily close to \(x\) that are not equal to \(x\).  
  		\item For any \(\epsilon > 0\), one can find \(y \in E\) such that \(y \in B_{\epsilon}(x)\) and \(x \not = y\)
  		\item There exists a sequence \((s_n)\) where each \(s_n \in E\) that converges to \(x\) and \(s_n\) is not constant valued (such as \(s_n = x\) for all n).
	\end{itemize}
The following two remarks are also important to keep in mind: 
\begin{itemize}
  \item It is possible for a limit point of \(E\) to not be an element of \(E\). For example, \(E = (0, 1) \subseteq \R\) has a limit point of 0 and a limit point of 1, but neither is contained within \(E\).
  \item It is possible that a point in \(E\) is \textbf{not} a limit point of E. For example, consider example \ref{exmp:limit_point_intro}. \(1 \in E\), but 1 is not a limit point of \(E\) since \(B_{\frac{1}{4}} \cap E = \{1\}\), and so since the intersection does not contain any values besides 1, we know that 1 is not a limit point of \(E\) by definition. 
\end{itemize}


\end{defn}
\begin{defn}{Closure}{closure}
Let \((S, d)\) be a metric space. Let \(E \subseteq S\). The \textit{closure} of \(E\) is written as \(\overline{E}\). 
\begin{equation*}
  \overline{E} := E \cup \{\textrm{limit points of E}\}
\end{equation*}

\end{defn}


%TODO Show that a corollary to this is that the smallest closed set containing A is its closure
\begin{thm}{}{}
Let \((S, d)\) be a metric space. Let \(E \subseteq S\). \newline 

\(E\) is closed \(\iff\) \(E = \overline{E}\) \newline 

For intuition of the forward direction, we notice that if a subset is closed but doesn't contain all of its limit points, we will be able to arrive at a contradiction by the same reasoning above. Thus, we will be able to show that a closed set must contain all of its limit points, and thus be equal to its closure.\newline

For the reverse direction, consider why we began our discussion of limit points. We saw that if there were any limit points of our subset that were not contained in our subset, then the complement of our subset would never be open, and so our subset would never be closed. However, if our subset contained all of its limit points, then the complement should be open, meaning that our subset will be closed. We will now formalize both of these proofs. \newline 

Forward Direction:
\begin{proof}
Assume \(E\) is a closed subset. Showing that \(E\) is equal to its closure is equivalent to showing that \(E\) contains all of its limit points. This is equivalent to showing that no limit points of \(E\) are in \(E^C\). Consider some limit point of \(E\), \(x \in S\). By definition of limit point, \(B_r(x) \cap E \not = \emptyset\) for all \(r > 0\). Since \(E\) is closed, \(E^C\) is open. Since \(E^C\) is open, it cannot contain the point \(x\) since \(B_r(x)\not\subseteq E^C\) for any \(r > 0\). Finally, \(x \not\in E^C \implies x \in E\). Since the limit point \(x\) is arbitrary, \(E\) contains all of its limit points.
\end{proof}
Reverse Direction: 
\begin{proof}
Assume \(E = \overline{E}\). To show \(E\) is closed, we must show \(E^C\) is open. Consider any \(x \in E^C\). Since \(E\) contains all of its limit points, we know that \(x\) is not a limit point of \(E\). This implies that \(B_r(x) \cap E = \emptyset \textrm{ or } \{x\}\) for some \(r > 0\). However, by assumption \(x \not\in E\), so \(B_r(x) \cap E = \emptyset\). Thus, \(B_r(x)\) must be contained entirely in \(E^C\), and so \(E^C\) is open. This implies \(E\) is closed.
\end{proof}
\end{thm}
\begin{exmp}{}{}
\(E = \{ \frac{1}{n} : n \in \N \} \cup \{0\} \subseteq \R\) is closed. We first notice that the only limit point of \(\{ \frac{1}{n} : n \in \N \}\) is 0, and so \(E\) contains all of its limit points. Thus, \(E\) is closed.
\end{exmp}



\subsection{Compact Subsets}
\begin{defn}{Open Covers and Finite Subcovers}{open_cover}
Let \((S, d)\) be a metric space. Let \(E \subseteq S\). Let \(\{u_i : i \in I\}\) be a collection of open subsets of \(S\).
\begin{itemize}
  \item \(\{u_i : i \in I\}\) is called an \textbf{open cover} of \(E\) if: 
  	\begin{equation*}
  		E \subseteq \bigcup_{i \in I}u_i
	\end{equation*}
	Equivalently, for any \(x \in E\), there exists \(i \in I\) such that \(x \in u_i\). In words: the union of the open sets contains all of \(E\).
  \item We say an open cover has a \textbf{finite subcover} if there exist finitely many \(i_1, i_2, \dots,i_n\) such that \(E \subseteq \bigcup_{k = 1}^n u_{i_k}\).
\end{itemize}

\end{defn}

\begin{defn}{Compact Subset}{compact_subset}
Let \((S, d)\) be a metric space. Let \(E \subseteq S\). \newline 
\(E\) is a \textbf{compact} subset if \textbf{\textit{every}} open cover of \(E\) has a \textit{finite subcover}.
	\newline 
	
	This definition should make it easy to show that a subset is not compact. We only have to find 1 example of an open cover of \(E\) that has \textbf{no} finite subcover to show that \(E\) is not compact.
\end{defn}
\begin{exmp}{}{}
\(\N = \{1, 2, 3, \dots\} \subseteq \R\) is not compact. We can show this by finding some open cover of \(\N\) that has no finite subcover. Since \(\N\) is of infinite size, it shouldn't be hard to create an open cover that requires infinitely many sets to cover all of \(\N\). For example, consider \(\{u_n = (n - \frac{1}{2}, n + \frac{1}{2}) : n \in N\}\). Each \(u_n\) is open, and for any \(x \in \N\), we know that \(x \in u_x\) by construction (if this is not immediately clear, try plugging in \(x\) to \(u_x\)). However, any finite collection of these \(u_n\) must have some largest value of \(n\). This means that \(u_{n+1}\) is not in the finite collection, and so \(n + 1\) is not in the union of the finite collection. Thus, no finite cover can exist for this open cover. 
\end{exmp}
\begin{exmp}{}{}
\(E = \{ \frac{1}{n} : n \in \N \} \cup \{0\} \subseteq \R\) is compact. Consider any open cover \(\{u_i : i \in I\}\). Since \(0 \in E\), there must exist \(i \in I\) such that \(0 \in u_i\). Since \(u_i\) is open, there exists \(r > 0\) such that \(B_r(0) \subseteq u_i\). Equivalently, \(u_i\) contains the set \(\{\frac{1}{n} : n > \frac{1}{r}\}\) (this follows from the fact that the values of \(\frac{1}{n}\) that are in \(B_r(0)\) satisfy \(d(\frac{1}{n}, 0) < r\) which means \(\frac{1}{n} < r\)). Note that \(\{ \frac{1}{n} : n \in \N \} = \{ \frac{1}{n} : 1 \leq n \leq \frac{1}{r} \} \cup \{ \frac{1}{n} : n > \frac{1}{r}\}\), and the term on the left of the union has finitely many values of n (since \(r > 0\) is finite), and the term on the right of the union takes on infinitely many values of n. We now want to make use of the fact that there are only finitely many values of \(\frac{1}{n}\) for \(1 \leq n \leq \frac{1}{r}\). Since there are finitely many values, and each value must be covered by at least one \(u_k\), it follows that there must be a finite number of \(u_k\) that cover these values. Finally, we can construct a finite subcover by taking each \(u_k\) along with the \(u_i\) we chose earlier. Thus, any open cover of \(E\) has a finite subcover, and so \(E\) is compact.
\end{exmp}

%TODO Try to explain intuition for why certain sets are compact and others are not, to prepare for the implication that compact sets are closed and bounded. Need to give more examples, and give commentary on each example.


%TODO Define sequentially compact
\begin{defn}{Sequentialy Compact}{sequentially_compact}
Let \((S, d)\) be a metric space. A subset \(E \subseteq S\) is called \textbf{sequentially compact} if for all sequences of points in \(E\), there is a subsequence that converges to a point in \(E\). \newline 

Note that this definition sounds somewhat similar to the definition of a closet subset, but we must be very careful to understand the difference. Earlier we showed that a subset \(E\)
 is closed if it contains all of its limit points. In other words, every sequence of points in \(E\) that converges (if any exist) must converge to a point in \(E\). However, the definition of sequentially compact is even stronger. It requires that \textbf{\textit{any}} sequence of points in \(E\) (whether or not the sequence converges is irrelevant) \textbf{MUST} have a subsequence that converges to a point in \(E\). Soon, we will prove that sequentially compact is a stronger statement than closedness: i.e. sequentially compact \(\implies\) closed.
 \end{defn}
\begin{exmp}{}{}
Consider \(E = (0, 1) \subseteq \R\). \(E\) is \textbf{not} sequentially compact. Consider the sequence \((1, \frac{1}{2}, \frac{1}{3}, \dots)\). Since this sequence converges to 0, all of its subsequences must also converge to 0. However, \(0 \not \in E\), so \(E\) is not sequentially compact.
\end{exmp}


See definition \ref{defn:metric_space_sequence_convergence}





\newpage
\section{Continuity}
\subsection{Continuous Functions}
First, let us define a few things to prevent confusion later:
\begin{defn}{Image of a Set}{}
Let \(f: X\to Y\) be a function from \(X\) to \(Y\) and \(u \subseteq X\). We call \(f(u)\) to be the \textbf{image} of \(u\) in \(Y\):
\begin{equation*}
  f(u) := \{f(x) \in Y : x \in u\}
\end{equation*}
\end{defn}

\begin{defn}{Pre Image of a Set}{}
Let \(f: X\to Y\) be a function from \(X\) to \(Y\) and \(u \subseteq Y\). We call \(f^{-1}(u)\) to be the \textbf{pre image} of \(u\) in \(X\):
\begin{equation*}
  f^{-1}(u) := \{x \in X : f(x) \in Y\}
\end{equation*}
\end{defn}


Now, we will define a few notions of continuity in terms of general metric spaces. Then, we will return to our focused analysis on the number system of \(\R\).
\begin{defn}{Continuity of a Function Between Metric Spaces}{}
Let \(f: (X,d_x) \to (Y, d_y)\) (A function from the metric space \((X, d_x)\) to the metric space \((Y, d_y)\)). \newline 

We say \(f\) is continuous at some \(x_0 \in X\) if for all sequences \((x_n) \in X\) where \(\limtoinf{n} x_n = x_0\):
\begin{equation*}
 \limtoinf{n} f(x_n) = f(x_0) \in Y 
\end{equation*}
 In other words, all sequences \((f(x_n))\) must converge to \(f(x_0)\). We call the sequence \((f(x_n))\) to be the \textbf{image} of the sequence \((x_n)\).\newline 
 
 We say that \(f\) is \textbf{continuous} if it's continuous at all \(x_0 \in X\).
\end{defn}
An equivalent formulation of the definition of continuity is given below:
\begin{thm}{Epsilon Delta Continuity}{epsilon_delta_continuity}
Let \(f: X \to Y\).
\begin{align*}
  &\textrm{\(f\) is continuous at \(x_0\in X\)} \\
  &\iff \\
   &\forall \epsilon > 0, \exists \delta > 0, s.t.:\\
    &d_x(x, x_0) < \delta \implies d_y(f(x), f(x_0)) < \epsilon
\end{align*}
Notice that we can rewrite the last line above in the language of open balls:
\begin{equation*}
  x \in B_\delta(x_0) \implies f(x) \in B_\epsilon(f(x_0))
\end{equation*}

For intuition, let us discuss the forward direction. We can easily show that the result holds by pursuing a proof by contradiction. In this case, we would assume that there exists some \(\epsilon > 0\) such that for all \(\delta > 0\), there will exist some \(x\) that is within \(\delta\) of \(x_0\), but the distance between \(f(x)\) and \(f(x_0)\) will be greater than \(\epsilon\). This means that for any \(B_{\delta}(x_0)\) we can find some \(x \in B_{\delta}(x_0)\) such that \(d_y(f(x), f(x_0)) \geq \epsilon\). Since \(\delta\) can be arbitrarily small, we know that we can find a convergent sequence \((x_n)\) which converges to \(x_0\) by having each \(x_n\) be an element of a smaller and smaller delta ball centered at \(x_0\) that allows \(d_y(f(x), f(x_0)) \geq \epsilon\). This means that the sequence \((f(x_n))\) cannot converge to \(f(x_0)\), which is a contradiction. We can formalize this thinking as well. \newline 

Forward Direction:
\begin{proof}
Assume \(f\) is continuous at \(x_0 \in X\). For sake of contradiction, we will assume the negation of the right side of the implication: there exists some \(\epsilon \geq 0\) such that for all \(\delta > 0\), there exists \(x \in X\) such that:
\begin{align*}
  &x \in B_{\delta}(x_0) \\
  &d_y(f(x), f(x_0)) \geq \epsilon
\end{align*}
Thus, we will create a sequence \((x_n)\) where \(x_n \in B_{\frac{1}{n}}(x_0)\) and \(d_y(f(x_n), f(x_0)) \geq \epsilon\) (these \(x_n\) are guaranteed to exist by the assumption). This sequence converges to \(x_0\), but \((f(x_n))\) does not converge to \(f(x_0)\) which is a contradiction to the definition of continuity.

\end{proof}

Reverse Direction:
\begin{proof}
Assume that:
\begin{align*}
   &\forall \epsilon > 0, \exists \delta > 0, s.t.:\\
    &d_x(x, x_0) < \delta \implies d_y(f(x), f(x_0)) < \epsilon
\end{align*}
In the language of open balls, we have \(x \in B_\delta(x_0) \implies f(x) \in B_\epsilon(f(x_0))\). Now, consider any sequence \((x_n) \in X\) that converges to \(x_0\). This means that there exists \(N > 0\) such that \(n > N \implies d_x(x_n, x_0) < \delta \iff x_n \in B_\delta(x_0) \implies f(x_n) \in B_\epsilon(f(x_0)) \implies d_y(f(x_n), f(x_0)) < \epsilon \implies (f(x_n))\) converges to \(f(x_0)\). Thus, any sequence \((x_n) \in X\) that converges to \(x_0\) must also have its image converge to \(f(x_0)\), and so the definition of continuity is satisfied.
\end{proof}


\end{thm}

\begin{exmp}{}{}
Let us see a quick corollary of theorem \ref{thm:epsilon_delta_continuity}: \newline 

Assume \(f: X \to Y\) is continuous. Consider any \(x_0 \in X\). Then for any \(\epsilon > 0\), there exists \(\delta > 0\) such that: 
\begin{equation*}
  f(B_\delta(x_0)) \subseteq B_\epsilon(f(x_0))
\end{equation*}

This is stating that the image of the delta ball will be contained in the epsilon ball. This is simply the result of applying theorem \ref{thm:epsilon_delta_continuity} to all \(x \in B_\delta(x_0)\).

\end{exmp}


Finally, we will show a third notion of continuity that is slightly more general. By this I mean that it is a valid definition of continuity in topological spaces, which are more general versions of metric spaces.

\begin{thm}{Continuity by Open Sets}{}
Consider some \(f:X\to Y\). The following statements are equivalent:
\begin{align*}
  &f \textrm{ is continuous}\\
  &\iff \\
  &u \underset{open}{\subseteq}Y \implies f^{-1}(u) \underset{open}{\subseteq} X.
\end{align*}

In otherwords, this theorem states that saying \(f\) is continuous is equivalent to saying that any open subset of \(Y\) must have an open pre image in \(X\).\newline 

The intuition for this proof follows from the epsilon delta definition of continuity. Namely, consider the formulation where we view the implication in terms of open balls. This allows the proof to flow naturally. For the forward direction, we want to show that for any open subset of \(Y\) and any \(x_0\) in the pre image of the subset, we can find a \(\delta\) such that the delta ball centered at \(x_0\) will be contained within the pre image of the subset. \newline 

Forward Direction:
\begin{proof}
Consider any open subset \(u \subseteq Y\). Consider any \(x \in f^{-1}(u)\). Since \(u\) is open, there exists \(\epsilon > 0\) such that \(B_\epsilon(f(x)) \subseteq u\). Since \(f\) is continuous, there exists \(\delta > 0\) such that \(f(B_\delta(x)) \subseteq B_\epsilon(f(x_0)) \subseteq u\). This directly implies that \(B_\delta(x) \subseteq f^{-1}(u)\), and thus, \(f^{-1}(u)\) is open.
\end{proof}
Reverse Direction:
\begin{proof}
Suppose that for any open subset \(u \subseteq Y\), it holds that \(f^{-1}(u)\) is an open subset of \(X\). Consider any sequence \((x_n) \in X\) that converges to \(x_0\). Since \(B_\epsilon(f(x_0))\) is an open subset of \(Y\), it must hold that \(f^{-1}(B_\epsilon(f(x_0)))\) is an open subset of \(X\). Thus, there must exist some \(\delta > 0\) such that \(B_\delta(x_0) \subseteq f^{-1}(B_\epsilon(f(x_0)))\). This directly implies that \(f(B_\delta(x_0)) \subseteq B_\epsilon(f(x_0))\), and thus \(f\) is continuous by theorem \ref{thm:epsilon_delta_continuity}.
\end{proof}
\end{thm}

\begin{thm}{Image of Compact Subset is Compact}{compact_image}
Let \(f: X \to Y\) be continuous function. 
\begin{equation*}
  E \subseteq X \textrm{ is compact} \implies f(E) \subseteq Y \textrm{ is compact}
\end{equation*}
Intuition: This is saying that if we can always cover \(E\) with a finite number of open sets in an open cover, then we should always be able to do the same for \(f(E)\). The proof will be rather simple, as we can relate an open cover of \(f(E)\) to an open cover of \(E\) using the third definition of continuity.


\begin{proof}
We know that if \(f\) is continuous, then for any open cover \(U\) of \(f(E)\), then \(f^{-1}(u\in U)\) will be an open subset of \(X\). Moreover, \(\{f^{-1}(u) : u \in U\}\) is an open cover of \(E\) since for any \(x \in E\), \(f(x) \in u_i\) for some \(i\). Thus, since \(E\) is compact, we can find a finite subcover of \(\{f^{-1}(u) : u \in U\}\). Call this subcover \(\{f^{-1}(u_i) : 1 \leq i \leq n\}\). Then, \(\{ u_i : 1 \leq i \leq n\}\) is a finite subcover of \(f(E)\) since for any \(f(x) \in f(E)\), there must be some \(x \in E\) such that for some \(i\), \(x \in f^{-1}(u_i)\). Thus \(f(x) \in u_i\). Therefore, for any open cover of \(f(E)\), we are able to construct a finite subcover.
\end{proof}

\end{thm}

\begin{thm}{Extreme Value Theorem for Compact Sets}{evt_compact}
If \(f: X \to \R\) is continuous, and \(E \subseteq X\) is compact, then:
\begin{itemize}
  \item \(f\) is bounded on \(E\) (there exists \(M > 0\) s.t. \(|f(x)| < M\) for all \(x \in E\))
  \item \(f\) attains its maximum and minimum on \(E\) (there exists \(x_1, x_2 \in E\) such that \(f(x_1) \leq f(x) \leq f(x_2)\) for all \(x \in E\)) 
\end{itemize}

Intuition: The first bullet point should be very obvious from our earlier discussion on compact sets. Namely, we proved that all compact sets are bounded. \newline 

The second bullet point will make use of the fact that the supremum and infimum exist for bounded sets in \(\R\), and moreover, since \(f(E)\) is compact by theorem \ref{thm:compact_image}, \(f(E)\) is also closed and bounded. Thus, the supremum and infimum of \(f(E)\) belong to \(f(E)\). This will allow us to find the values of \(x \in E\) that correspond to the supremum and infimum, and the claim will easily follow. \newline 
\begin{proof}
The first bullet point follows from theorem \ref{thm:compact_implies_bounded}. \newline 

Since \(f(E)\) is also closed by theorem \ref{thm:compact_implies_closed}, it follows that \(\sup f(E) \in f(E)\) and \(\inf f(E) \in f(E)\) (if \(\sup f(E)\) and \(\inf f(E)\) were not in \(f(E)\), then they would have to be limit points of \(f(E)\), but since \(f(E)\) is closed it must contain all of its limit points). Thus, there exist \(x_1,x_2 \in E\) such that:
\begin{itemize}
  \item \(\sup f(E) = f(x_2)\)
  \item \(\inf f(E) = f(x_1)\)
\end{itemize}
Thus, \(f(x_1) \leq f(x) \leq f(x_2)\) for all \(x \in E\).
\end{proof}


\end{thm}
\begin{exmp}{}{}
We will often use the extreme value theorem in real analysis when dealing with some interval \([a,b]\). We can use theorem \ref{thm:evt_compact} along with the fact that \([a,b]\subseteq \R\) is compact to show that any continuous \(f: [a,b] \to \R\) will attain its max and min on \([a,b]\).

\end{exmp}

\begin{thm}{Image of Connected Subset is Connected}{connected_image}
Let \(f:X\to Y\) be continuous. Then:
\begin{equation*}
  E \subseteq X \textrm{ is connected} \implies f(E) \subseteq Y \textrm{ is connected}
\end{equation*}
See definition \ref{defn:connected} to review definition of connected. \newline 

\begin{proof}
Assume for sake of contradiction that \(f(E) \subseteq Y\) is disconnected. Then, there exist \(u_1,u_2 \opensub Y\) that separate \(f(E)\). Since \(f\) is continuous, \(A := f^{-1}(u_1)\) and \(B := f^{-1}(u_2)\) are open subsets of \(E\). Now, note the following:
\begin{itemize}
  \item \(E \cap A \neq \emptyset\) and \(E \cap B \neq \emptyset\) because \(u_1\) and \(u_2\) are non-empty.
  \item For any \(x \in E\), \(f(x) \in f(E)\), and thus \(f(x) \in u_1 \cup u_2\). This implies \(x \in f^{-1}(u_1 \cup u_2) = A \cup B\), and so \(E \subseteq A \cup B\)
  \item Consider any \(x \in A\). Thus, \(f(x) \in u_1\).  Since \(u_1\) and \(u_2\) are disjoint, \(f(x) \not\in u_2\). Thus, \(x \not \in B\). This holds for all \(x \in A\), and thus \(E \cap A \cap B = \emptyset\)
\end{itemize}
These three results in conjunction imply that \(A,B\) separates \(E\), and thus imply that \(E\) is disconnected, which is a contradiction.

\end{proof}

\end{thm}

\begin{thm}{Intermediate Value Theorem}{IVT}
Let \(I \subseteq \R\) be an interval. Let \(f: I \to \R\) be a continuous function. Then, for all \(a,b \in I, a<b,\) and for all \(y\in (f(a), f(b))\), there exists \(x \in [a,b]\) such that \(f(x) = y\). \newline 

\begin{proof}
We saw earlier that an interval \([a,b] \in \R\) is connected. Thus, since \(f\) is continuous, \(f([a,b])\) is also a connected subset of \(\R\), and so \(f([a,b])\) is an interval as well. Since \(f(a), f(b) \in f([a,b])\), \(y \in f([a,b])\) as well. This implies that there exists \(x \in [a,b]\) such that \(f(x) = y\).
\end{proof}


\end{thm}











\subsection{Uniform Continuity}
\subsection{Convergence of Sequences of Functions}
The weakest form of convergence that we will analyze in this course is pointwise convergence.
\begin{defn}{Pointwise Convergence}{pointwise_convergence}
Let \(X\) be a set , and \((f_n)\) be a sequence of functions \(f_n: X \to \R\). We claim that \((f_n) \to f\) converges pointwisely to a function \(f: X \to \R\) if \(\forall x_0 \in X\), we have \(\lim_{n \to \infty} f_n(x_0) = f(x_0)\) We can also write this equivalently as (insert epsilon definition of pointwise convergence).
\end{defn}
Note that a sequence of continuous functions can converge pointwisely to a non continuous function (consider \((f_n = x^n)\)). (Insert triangle function example with constant area of 1/2 that converges to f(x) = 0, but does not converge in integral). We can now define a stronger notion of convergence that deals with the issues of failure to converge in continuity and in integration. 
\begin{defn}{Uniform Convergence}{function_uniform_convergence}
We claim that \((f_n) \to f\) converges uniformly to a function \(f: X \to \R\) if \(\forall \epsilon > 0,\exists N > 0\) such that \(\forall x_0 \in X,\forall n > N, |f_n(x_0) - f(x_0)| < \epsilon\). The key difference between this definition of convergence and the pointwise definition of convergence is that uniform convergence requires \(N\) to be dependent only on \(\epsilon\), where as \(N\) can be dependent on both \(\epsilon\) and \(x_0\) in the definition of pointwise convergence.
\end{defn}
Let us now see how exactly uniform convergence avoids some of the issues with pointwise convergence.
(Insert theorem that a sequence of continuous functions that converge uniformly implies that their convergent function is also continuous)
\begin{thm}{}{continuous_uniform_implies_continuous}
If \((f_n)\) is a sequence of continuous functions on some set \(X\), and \((f_n) \to f\) uniformly, then \(f\) is continuous on \(X\).
\newline
\begin{proof}
Since \((f_n)\to f\) uniformly, we know that \(\exists N\) s.t. 
c\begin{equation*}
n > N \implies |f(x) - f_n(x)| < \frac{\epsilon}{3}
\end{equation*}

Now, we pick some \(n>N\). Since \(f_n\) is continuous, we know that \(\exists \delta > 0\) s.t. 

\begin{equation*}
|x - x_0| < \delta \implies |f_n(x) - f_n(x_0)| < \frac{\epsilon}{3}
\end{equation*}

\(\forall x_0,x \in X, \forall \epsilon > 0,\)
\newline
\(|x - x_0| < \delta \implies\) 
\begin{align*}
|f(x) - f(x_0)| &\leq |f(x) - f_n(x)| + |f_n(x) - f_n(x_0)| + |f_n(x_0) - f(x_0)| \\
		   &< \frac{\epsilon}{3} + \frac{\epsilon}{3} + \frac{\epsilon}{3} \\
		   &= \epsilon
\end{align*}
Thus, \(f\) is continuous on \(X\).




\end{proof}
\end{thm}
(Insert theorem that if we have a sequence of continuous functions that converge to f uniformly, the limit of the integral of \(f_n\) is the same as the integral of f)







\end{document}